% apl.tex

\documentclass[12pt,fleqn]{article}
%\usepackage[cp437de]{inputenc}
\usepackage{times}
\usepackage{graphicx}
\graphicspath{{images/}{../images/}}
\usepackage{latexsym}
\usepackage{setspace} 

\usepackage{subfiles}
\usepackage{hyperref}	% Allows for clickable references

\doublespacing

\sloppy
\hfuzz            3.0pt
\topmargin       -0.5cm
%\oddsidemargin    1.6cm
%\oddsidemargin    1.2cm
\oddsidemargin    2.0cm
\evensidemargin   1.5cm
%\evensidemargin   1.1cm
\headsep          1.0cm
%\textwidth       16.0cm
%\textheight      23.0cm
\textwidth       16.5cm
\textheight      22.0cm
\hoffset=        -1.5cm
\parskip=         1.0ex
\parindent=       1.0cm
\partopsep=       0.0ex
\parsep=          0.3ex
\topsep=          0.0ex
\itemsep=         0.5ex
\mathindent       0.0em
\hyphenpenalty 1000
\clubpenalty  10000
\widowpenalty 10000

\renewcommand{\topfraction}         {1.0}
\renewcommand{\bottomfraction}      {0.9}
\renewcommand{\textfraction}        {0.0}
\renewcommand{\floatpagefraction}   {0.8}
\renewcommand{\dbltopfraction}      {0.7}
\renewcommand{\dblfloatpagefraction}{0.5}
\renewcommand{\arraystretch}				{2.00}

\setcounter{secnumdepth} {0}
\setcounter{tocdepth}    {3}
\setcounter{topnumber}   {9}
\setcounter{bottomnumber}{9}
\setcounter{totalnumber} {9}
\setcounter{dbltopnumber}{9}
\def\figurename{\small Figure}
\def\tablename{\small Table}
\def\d{{\rm d}}
\def\e{{\rm e}}
\def\j{{\rm j}}
\def\hb{\hfil\break}
\def\ve{\vfill\eject}
\def\lk{[}
\def\rk{]}
\def\mo{\mu_0}
\def\la{\lambda_{100}}
\def\lb{\lambda_{111}}
\def\ni{\vec n_i}
\def\nk{\vec n_k}
\def\Nd{\underline{N_d}}
\def\sgn{\mbox{sgn}}

\def\be{\begin{equation}}
\def\ee{\end{equation}}
\def\bea{\begin{eqnarray}}
\def\eea{\end{eqnarray}}
\def\bc{\begin{center}}
\def\ec{\end{center}}
\def\bn{\begin{enumerate}}
\def\en{\end{enumerate}}
\def\bi{\begin{itemize}}
\def\ei{\end{itemize}}

% ***************************************************************************

\begin{document}

%\pagestyle{headings}
%\pagestyle{myheadings}
\markboth{Draft 1.0}{Draft 1.0}

% \baselineskip=2\normalbaselineskip

\thispagestyle{empty}

\title
{\bf
   Effects of microstructure and anisotropy on hysteresis in two theoretical
    models describing the behavior of ferromagnetic materials at both high
                            and low magnetic fields
\\[0.5cm]
}
\author
{
               H. Hauser \footnote{Deceased}
\\
\normalsize
               Vienna University of Technology, Vienna, Austria
\\[0.5cm]
               Y. Melikhov and D. C. Jiles
\\
\normalsize
               Cardiff University, Cardiff, United Kingdom
\\
}
\date{} %\today}

\maketitle
\thispagestyle{empty}

\cleardoublepage

\paragraph{Abstract:}

Two recent theoretical hysteresis models (Jiles-Atherton model and Energetic model) are examined with respect to their capability to describe the dependence of the magnetization on magnetic field, microstructure and anisotropy. It is shown that the classical Rayleigh Law for behavior of magnetization at low fields, and the widely used Stoner-Wohlfarth theory of domain magnetization rotation in non-interacting magnetic single domain particles, can be considered as limiting cases of a more general theoretical treatment of hysteresis in ferromagnetism.

\paragraph{Keywords:} Magnetism, non-linear behavior, hysteresis, microstructure, anisotropy, mechanical stress, Rayleigh region, Stoner-Wohlfarth theory

\cleardoublepage

\subfile{sections/01_introduction.tex}

\subfile{sections/02_Rayleigh_Law.tex}

\subfile{sections/03_Micromagnetics.tex}

\subfile{sections/04_Anisotropy.tex}

\subfile{sections/05_Hysteretic_Theories.tex}

\subfile{sections/06_Conclusions.tex}


\begin{thebibliography}{99}
\addcontentsline{toc}{section}{References}

\bibitem{Bertotti}
G. Bertotti, {\em Hysteresis in Magnetism} (Academic Press, London, 1998).

\bibitem{DellaTorre}
E. Della Torre, {\em Magnetic Hysteresis} (IEEE Press, Piscataway, 1999).

\bibitem{Ivanyi}
A. Iv\'anyi, {\em Hysteresis Models in Electromagnetic Computation}
(Akad\'emiai Kiad\'o, Budapest, 1997).

\bibitem{Mayergoyz}
I. D. Mayergoyz, {\em Mathematical Models of Hysteresis} (Springer, New York,
1991).

\bibitem{Jiles}
D. C.\ Jiles and D. L.\ Atherton,
J.\ Magn.\ Magn.\ Mater.\ {\bf 61}, 48 (1986).

\bibitem{JAP75}
H. Hauser,
J. Appl.\ Phys.\ {\bf 75}, 2584 (1994).

\bibitem{Jiles07}
D. C. Jiles and Y. Melikhov,
in {\em Handbook of Magnetism and Advanced Magnetic Materials, vol. 2: Micromagnetism},
Eds. H. Kronm\"uller and S. Parkin (John Wiley \& Sons, Hoboken, 2007).

\bibitem{JAP96}
H. Hauser,
J. Appl.\ Phys.\ {\bf 96}, 2753 (2004) and
J. Appl.\ Phys.\ {\bf 97}, 099901-1 (2005).

\bibitem{Kron}
H. Kronm\"uller and M. F\"ahnle, {\em Micromagnetism and the Microstructure of
Ferromagnetic Solids} (Cambridge University Press, Cambridge, 2003) pp.~71--173.

\bibitem{TMAG}
H. Hauser, D. C. Jiles, Y. Melikhov, and L. Li,
unpublished.


\bibitem{Kersten}
M. Kersten, {\em Probleme der Technischen Magnetisierungskurve} (Springer,
Berlin, 1938), pp.\ 42--72.

\bibitem{Kondorsky}
E. Kondorsky, Z. Phys.\ Sov.\ {\bf 11}, 597 (1937).

\bibitem{Appino}
C. Appino, F. Fiorillo, and A. Maraner,
IEEE Trans.\ Magn.\ {\bf 29}, 3469 (1993).

\bibitem{Fujimori}
H. Fujimori, in {\em Amorphous Metallic Alloys}, Ed. F. E. Luborsky
(Butterworths, London, 1983) pp.~300--316.

\bibitem{Stoner48}
E. C. Stoner, and E. P. Wohlfarth, 
Phil.\ Trans.\ R.\ Soc.\  {\bf 240}, 599%-642
(1948).

\bibitem{JMMM}
H. Hauser, D. C. Jiles, Y. Melikhov, L. Li, and R. Gr\"ossinger,
J. Magn.\ Magn.\ Mater.\ {\bf 300}, 273 (2006).

\bibitem{Rayleigh}
J.W. Rayleigh,
Phil.\ Mag. {\bf 5}, 225 (1887).

\end{thebibliography}

\newpage

\begin{table}[h]
\renewcommand{\thetable}{\Roman{table}}
\caption{JAM and EM coefficients of Fe-C series \cite{TMAG}.}
\label{C}
\bc
\begin{tabular}{lrrrr}
\hline
wt\%C                  &   0.00 &   0.24 &   0.47 &   0.74 \\
\hline
$M_s$ [MA/m]           &  1.700 &  1.655 &  1.609 &  1.555 \\
$a$ [kA/m]             &  2.623 &  2.427 &  2.306 &  1.429 \\
$\alpha$ [$10^{-3}$]   &  4.280 &  3.901 &  3.447 &  1.849 \\
$c$ [$10^{-1}$]        &  2.002 &  1.213 &  0.743 &  0.509 \\
$k_{\rm J}$ [A/m]      &    245 &    532 &    827 &    958 \\
\hline
$k_{\rm H}$ [J/m$^{3}$]&    422 &    980 &   1560 &   1810 \\
$q$ [1]                &   13.7 &   16.9 &   22.6 &   29.5 \\
$N_i$ [$10^{-4}$]      &   1.70 &   4.99 &   8.54 &   9.08 \\
\hline
\end{tabular}
\ec
\end{table}

\newpage

\listoffigures

\newpage

%\subfile{sections/10_images.tex}

\end{document}
