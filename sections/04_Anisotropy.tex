\documentclass[../main.tex]{subfiles}
 
\begin{document}

\section{Anisotropy}

\subsection{Coefficient relations to anisotropy at weak fields}

Additionally, we can also relate some coefficients of the JAM and EM to
magnetomechanical parameters of the material. The following
considerations deal with elastic deformation only, plastic deformation
leads to additional defects that could be treated as described above.

The coercivity of many bulk soft magnetic materials is determined by inner
stress $\sigma_i$ and isotropic magnetostriction constant $\lambda_s$ as
\begin{equation}
  H_c = {\lambda_s \sigma_i \over \mu_0 M_s}
    \label{Hc}
\end{equation}
which has been discussed earlier by Kersten \cite{Kersten}. The inner stress
$\sigma_i$ is mainly proportional to Young's modulus $E_Y$ as
$\sigma_i\propto\lambda_sE_Y$ and we can define
\begin{equation}
  c_{k} = {\sigma_i\over \lambda_s E_Y}
  \label{cks}
\end{equation}
with the microscopic constant $c_{k}>1$ ($c_{k}\approx 1$ in carefully
stress relief annealed materials).
The initial susceptibility can be related to the effective anisotropy
in the case of magnetization rotation \cite{Kondorsky},
\begin{equation}
  \chi_0 = {1\over c_q}\,{\mu_0 M_s^2\over \Delta w} \ ,
  \label{cq}
\end{equation}
where $c_q$ is dimensionless proportionality constant and $\Delta w$ represents
the anisotropy energy maximum to be overcome during
magnetization reversal. This is mainly the case in amorphous materials, where
the initial susceptibility contribution of DWDs is small \cite{Appino}.
But the domain structure is strongly effected by the anisotropy variation due to
stress and a decreasing $\chi_0$ is expected also in DWD dominated magnetization
processes.


\subsection{Anisotropy and EM relations:}

Using Eqs.~(\ref{Hc}) and (\ref{cks}), the coefficient $k_H$ is
\begin{equation}
  k_{\rm H} = c_{k} \lambda_s^2 E_Y \ .
  \label{ks}
\end{equation}
Usually, $E_Y$ depends only little on $\sigma$ and an applied stress will only
slightly modify $k_H$ and $H_c$; we can then consider the product $c_{k}E_Y$ to be
constant.

If we simplify Eq.~(\ref{xEM}) with $k_H\gg\mu_0\,M_s\,N_i$, which is valid for
most materials (see, for example, Tab.~\ref{C}), by combining
Eqs.~(\ref{xEM}) and ~(\ref{cq}) and using
\begin{equation}
	\Delta w = \left|{K_u-{3\over 2}\,(1+\nu_P)\,\lambda_s\,\sigma}\right|
\end{equation}
we can calculate the coefficient $q$
\begin{equation}
  q = c_q\,{\Delta w \over \mu_0 M_s^2} = c_q \left|{K_u-{3\over 2}\,(1+\nu_P)\,\lambda_s\,\sigma
    \over c_{k} \lambda_s^2 E_Y}\right|
  \label{qs}
\end{equation}
where $\nu_P$ is Poisson ratio. A constant $q_0$ could be added to the r.h.s.
of the Eq.~(\ref{qs}) which considers that $\chi_0$ is not exclusively determined by
magnetization rotation. $q=0$ would lead to the anhysteretic curve. $K_u$
represents a small uniaxial anisotropy
\cite{Appino},
for instance arising from ordering effects \cite{Fujimori} which is responsible
for a maximum $\chi_0$ at $\sigma\ne 0$.


\subsection{Anisotropy and JAM relations:}

Considering Eqs.~(\ref{Hc}) and (\ref{cks}) are related to pinning only
($H_c\approx k_{\rm J}$), we find
\begin{equation}
  k_{\rm J} = c_k\,{\lambda_s^2\,E_Y\over \mu_0\,M_s} \ .
\end{equation}
If in Eq.~(\ref{X0J}) we let $a$ and $\alpha$ to be proportional to $\Delta w$ \cite{JMMM},
we see that $\chi_0$ is decreasing with increasing stress. The coefficient $c$ will thus not significantly depend on $\sigma$ in most cases.

\cleardoublepage

\end{document}