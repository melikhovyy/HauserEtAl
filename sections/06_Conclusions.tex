\documentclass[../main.tex]{subfiles}
 
\begin{document}

\section{Conclusions}
The results of this investigation have shown that the Rayleigh law at low fields and Stoner-Wohlfarth coherent rotation can both be described as limiting cases of a more general hysteresis model which encompasses both low and high field behavior. Through the more general hysteresis model it is possible to include additional features, for example a smooth  approach to saturation at $\varphi=0$ and $\varphi=\pi/2$, $H_c>0$ in the hard axis direction, $H_c<H_k$ in the easy axis direction, etc. Furthermore, a correlation to micromagnetic modeling showed the dependence of the model coefficients on the defect density and thus the ability to describe the effects of degradation by hysteresis measurements. The predictions of the Rayleigh Law and Stoner Wohlfarth magnetization curves may therefore be considered as a special case of a more generalized hysteresis model that can be described by the JAM and EM.

The analysis revealed that the coefficients of the anhysteretic functions of the JAM and EM depend approximately linearly on the effective anisotropy. This can be utilized to build relationships between these phenomenological coefficients and with the spontaneous magnetization and the constants of anisotropy and magnetostriction.

\section{Acknowledgments}
Financial support was provided by the Fulbright Commission, US Department of State, Grant No.~G-1-00005, which is gratefully acknowledged. This work was also supported in part by the U.S. Department of Energy, Office of Basic Energy Science, Materials Science Division. The research was performed at Ames Laboratory. Ames Laboratory is operated for the US Department of Energy by Iowa State University under contract number W-7405-ENG-82.

\cleardoublepage

\end{document}