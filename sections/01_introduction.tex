\documentclass[../main.tex]{subfiles}
 
\begin{document}

\section{Introduction}

The theoretical description of non-linear, hysteretic processes in general, and ferromagnetic hysteresis in particular, is known to be a difficult problem. In the case of ferromagnetism this is due to the multiplicity of processes, both reversible and irreversible, that take place simultaneously within a ferromagnet under the action of a magnetic field.

Successful descriptions of the behavior of magnetic materials under the influence of the applied field and its history by hysteresis modeling have great impact to the field of magnetism \cite{Bertotti, DellaTorre, Ivanyi, Mayergoyz}. In addition it would be very useful to predict the changes of the magnetic properties due to other physical quantities, like applied and residual stresses, fatigue, temperature, or irradiation, for example. Moreover, applications require the integration of the model into system design software with sufficient simplicity that will allow fast computation and efficient coefficient identification strategies.

In this work two recent hysteresis models: Jiles-Atherton model (JAM) by Jiles and Atherton \cite{Jiles} and Energetic model (EM) by Hauser \cite{JAP75}, were studied in order to relate their model parameters with the microstructure and anisotropy, as well as to study these models in some limiting cases. The detailed descriptions of these two models can be found elsewhere \cite{Jiles, JAP75}. For both models we were using simplified anhysteretic functions: either a linear or hyperbolic tangent, which is well-known in situations where the magnetic moments are constrained to lie along a single axis ("spin up" or "spin down"). This is referred to as the one dimensional case. Also, working with two and three dimensional cases is possible: the former leading to a series solution for the anhysteretic function, the latter leading to the Langevin function \cite{Jiles07}. The indices J and H are to distinguish between the $k$ coefficients in the  JAM and EM, respectively.

\cleardoublepage

\end{document}